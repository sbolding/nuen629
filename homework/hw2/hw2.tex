% --------------------------------------------------------------
% This is all preamble stuff that you don't have to worry about.
% Head down to where it says "Start here"
% --------------------------------------------------------------
 
\documentclass[12pt]{article}

\usepackage{courier}
\usepackage{color}
\usepackage{listings}
\usepackage[square,numbers]{natbib}
\usepackage{tabls}
\usepackage{graphicx}
\usepackage{subcaption}
\usepackage{pdfpages}
\usepackage{mathtools}

\definecolor{dkgreen}{rgb}{0,0.6,0}
\definecolor{gray}{rgb}{0.5,0.5,0.5}




\lstset{language=python,
   basicstyle=\ttfamily,
   keywordstyle=\color{blue},
   commentstyle=\color{dkgreen},
   stringstyle=\color{red},
   numbers=left,
   numberstyle=\tiny\color{gray},
   stepnumber=1,
   numbersep=10pt,
   backgroundcolor=\color{white},
   tabsize=4,
   showspaces=false,
   showstringspaces=false}
 
\usepackage[margin=1in]{geometry} 
\usepackage{amsmath,amsthm,amssymb}
\usepackage{verbatim}
\usepackage{algpseudocode,algorithm}
\usepackage{setspace}

\newcommand{\ihat}{\ensuremath{\hat{\textbf{\i}}}}
\newcommand{\jhat}{\ensuremath{\hat{\textbf{\j}}}}
\newcommand{\lline}{\noindent\makebox[\linewidth]{\rule{\textwidth}{0.4pt}}}
\newcommand{\N}{\mathbb{N}}
\newcommand{\Z}{\mathbb{Z}}
\newcommand{\deriv}[2]{\frac{\mathrm{d} #1}{\mathrm{d} #2}}
\newcommand{\pderiv}[2]{\frac{\partial #1}{\partial #2}}
\newcommand{\bx}{\mathbf{X}}
\newcommand{\ba}{\mathbf{A}}
\renewcommand{\d}{\mathrm{d}}
\newcommand{\A}{\frac{(1-\alpha)}{2(1+\alpha)}}
\newcommand{\upl}{u_{\text{plane}}}
\newcommand{\upt}{u_{\text{point}}}
\newcommand{\D}{\Delta}
\newcommand{\ra}{\rightarrow}
\renewcommand{\SS}{\State}
 
\newenvironment{theorem}[2][Theorem]{\begin{trivlist}
\item[\hskip \labelsep {\bfseries #1}\hskip \labelsep {\bfseries #2.}]}{\end{trivlist}}
\newenvironment{lemma}[2][Lemma]{\begin{trivlist}
\item[\hskip \labelsep {\bfseries #1}\hskip \labelsep {\bfseries #2.}]}{\end{trivlist}}
\newenvironment{exercise}[2][Exercise]{\begin{trivlist}
\item[\hskip \labelsep {\bfseries #1}\hskip \labelsep {\bfseries #2.}]}{\end{trivlist}}
\newenvironment{problem}[2][Problem]{\begin{trivlist}
\item[\hskip \labelsep {\bfseries #1}\hskip \labelsep {\bfseries #2:}]\hspace{0.3in}\newline\newline}{\end{trivlist}}
\newenvironment{question}[2][Question]{\begin{trivlist}
\item[\hskip \labelsep {\bfseries #1}\hskip \labelsep {\bfseries #2.}]}{\end{trivlist}}
\newenvironment{corollary}[2][Corollary]{\begin{trivlist}
\item[\hskip \labelsep {\bfseries #1}\hskip \labelsep {\bfseries #2.} ]}{\end{trivlist}}
\newenvironment{problem*}[1][Problem]{\begin{trivlist}
\item[\hskip \labelsep {\bfseries #1} {\hspace{-0.2em}\bfseries:}]}{\end{trivlist}}
\newenvironment{solution}[1][Solution]{\begin{trivlist}
\item[\hskip \labelsep {\bfseries #1} {\hspace{-0.2em}\bfseries:}]\hspace{0.3in}\newline}{\end{trivlist}}
\newenvironment{solnum}[2][Solution]{\begin{trivlist}
\item[\hskip \labelsep {\bfseries #1}\hskip \labelsep {\bfseries #2:}]\hspace{0.3in}\newline\newline}{\end{trivlist}}
\newcommand{\iso}[2]{\ensuremath{^{#2}\text{#1}}}
\newcommand{\nubar}{\ensuremath{\overline{\nu}}}
 
\begin{document}
 
% --------------------------------------------------------------
%                         Start here
% --------------------------------------------------------------
 
\title{Homework 2}%replace X with the appropriate number
\author{Simon Bolding\\ %replace with your name
NUEN 629} %if necessary, replace with your course title
 
\maketitle

\clearpage

\includepdf{Homework2.pdf}

\begin{solnum}{1}
    
Several approximations were made to simplify the process.  First, graphite is
approximated as elemental Carbon, with molar mass 12.0107 (g/mol).  This was done because there is not a human
friendly form of the graphite cross sections available on NNDC and elemental Carbon
will have similar scattering properties, except for at low energies where diffraction is
possible.  A plot of the elastic and total cross sections for elemental Carbon from
NNDC are given below.
\begin{figure}[h!]
\centering
\includegraphics[width=0.5\textwidth]{carb_cx.pdf}
\end{figure}


The ratio of 150:1 for graphite to natural uranium is assumed to be an of
atomic ratio.  Natural uranium is
taken to be 0.72\% $^{235}$U and the remainder $^{238}$U, by atom percentage. 
The total cross section
is assumed to only consist of elastic scattering, fission, and removal events.

We follow a similar procedure to the one in lab.  For an infinite medium, with
fine-group cross sections, the balance equation becomes
\begin{multline}
    N^{U}\sum_{j} \gamma_j \sigma_t^j(E) \psi(\mu,E) = N^{U}\sum_{j} \gamma_j
    \frac{1}{2} \int\limits_0^\infty \d E' \sigma_s^j (E'\ra E)\phi(E') + \\
    N^U\frac{\chi(E)}{2k} \int_0^\infty\d E' \left(\gamma_{238}\nubar\sigma_f^{238} +
    \gamma_{235}\nubar\sigma_f^{235}\right)\phi(E'),
\end{multline}
where $j$ indicates the $j$-th isotope, $N^U$ is the atom density of natural uranium
in the system, and $\gamma_j$ is 150, 0.9928, and 0.0072 for Carbon, \iso{U}{238},
and \iso{U}{235}, respectively.  It is assumed $\chi(E)$ is the same for \iso{U}{238}
and \iso{U}{235}, given by the Watt spectrum from class
\begin{equation}
    \chi(E) = 0.4865 \sinh(\sqrt{2E})e^{-E}.
\end{equation}
We now simplify by normalizing such that the energy integrated fission source has a
magnitude of 1.  We also assume all scattering events result in the average
scattering energy loss, which, assuming isotropic scattering in the center of mass
frame, gives an average outgoing energy of 
\begin{equation}
    \langle E\rangle = \frac{A^2 + 1}{(A+1)^2}E'
\end{equation}
in the lab frame.  With this simplificiation, only a particular $E'$ governed by the above equation can scatter
into $E$, so the the elastic scattering source for the $j$-th term in the summation can be simplified as
\begin{align}
    \int_0^\infty \d E' \sigma_s^j(E')P(E'\ra E)\phi(E') &= \int_0^\infty \d
    E'\sigma_s^j(E')\delta\left(E' - E \frac{E'}{\langle E \rangle}\right) \phi(E') \\
    &= \sigma_s^j\left(\frac{(A+1)^2}{A^2+1}E\right) \phi\left(\frac{(A+1)^2}{A^2+1}E\right) \\
\end{align}
where $A$ is the atomic mass number for the $j$-th isotope, approximated as 12.0107 for elemental carbon.
Substituting back into the original equation and integrating over angle gives the
final equation for the scalar flux as
\begin{equation}
    \sum_{j} \gamma_j \sigma_t^j(E) \phi(E) = \sum_{j} \gamma_j \sigma_s^j \left(\frac{(A+1)^2}{A^2+1}E\right) \phi\left(\frac{(A+1)^2}{A^2+1}E\right) 
   + {\chi(E)} .
\end{equation}
We solve this equation with the Jacobi iteration
\begin{equation}
    \sum_{j} \gamma_j \sigma_t^j(E) \phi^{(k)}(E) = \sum_{j} \gamma_j \sigma_s^j
    \left(\frac{(A+1)^2}{A^2+1}E\right) \phi^{(k-1)}\left(\frac{(A+1)^2}{A^2+1}E\right) 
   + {\chi(E)} .
\end{equation}
with an initial guess of $\phi^{(0)}(E)=0$.
To approximate the continuous energy cross sections and $\phi(E)$ we simply evaluate the
above iteration at each of the energy points of the fine group cross sections. The
points are defined using the union of the total cross section energy grids of all
isotopes.  A linear interpolation (python interp1D default interpolation) is used between energy points when one
cross section is coarser than others. For evaluation above the maximum energy for a
given cross section, the value of the cross section at the maximum energy is used.
All values of cross sections and flux above 20 MeV were ignored.  

Once $\phi(E)$ is obtained, the collapsed cross sections are computed as
\begin{equation}
    \sigma_{n,g} = \frac{\int_{E_g}^{E_{g-1}} \d E  \sum_{j} \gamma_j \sigma_n^j(E)
\sum\phi(E)}{\int_{E_g}^{E_{g-1}} \d E \phi(E)}
\end{equation}
for the $g$-th group and $n$-th reaction type. The integral is approximated with
midpoint quadrature.


In Fig.~\ref{1} the obtained solution for the fully converged spectrum is plotted against the initial
uncollided spectrum, where
both are normalized relative to the integral of the converged $\phi(E)$. As shown,
the spectrum demonstrates significant moderation to lower energies due to the large amount of graphite
present. The collapsed group cross sections are given in Table~\ref{cx}.
The python script used to compute the answers is given at the end of the assignment.
\begin{figure}[htb]
\centering
\includegraphics[width=0.6\textwidth]{uc_spect.pdf}
\caption{\label{1}Comparison of scattering corrected and uncollided spectrum for
infinite-medium mix of natural Uranium and elemental Carbon}
\end{figure}
\begin{table}
    \centering
    \caption{3 Group cross section for 150:1 Carbon Uranium mix}
    \begin{tabular}{|ccc|}\hline
        Group & $\sigma_{t,g}$ (b) & $\sigma_{s,g}$ (b) \\ \hline
         0    &  395.1   & 391.4 \\ 
         1    &  697.2   & 695.9 \\
         2    &  732.8   & 729.6 \\ \hline
     \end{tabular}
\end{table}






\end{solnum}

\clearpage

\begin{solnum}{2}

First, we need an atom density of the hydrogen.  Based on ideal gas law, the density
scales proportional to tempurature.  At 1 atm, the density of hydrogen at room
temperature is roughly 9.E-04 g cm$^{-3}$.  Scaling by 30, gives a density of 0.003
g cm$^{-3}$.  This is in agreement to 1 significant digit with an online calculator that takes into account
compressibility of hydrogen at 30 atm and room temperature.  This gives a
corresponding atom density of hydrogen as
\begin{equation}
    N^{H} = 0.003 \frac{\text{g}}{\text{cm}^3}\frac{1\text{ mol H}_2}{2.02\text {g
    H}_2}\frac{2 H}{1 H_2}\frac{0.60221\text{ atoms cm}^2}{\text{b}} \approx
    0.002\;\frac{\text{H atoms}}{\text{b--cm}}
\end{equation}

As an initial approximation, I assumed that within the semi-infinite Hydrogen medium, several
mfp away from the source, the sphere of \iso{U}{235} will appear as an isotropic
boundary source to a 1D, semi-infinite medium.  Following the notes, the general multigroup
1D transport equation we will be solving, in the Hydrogen, will have the form
\begin{equation}\label{te_orig}
    \mu\pderiv{\psi_g(z,\mu)}{z} +\hat\Sigma_{tg}\psi_g = \sum_{l=0}^\infty
    \frac{2l+1}{2} P_l(\mu) \sum_{g'=0}^{G-1}\left[\Sigma_{slg'\ra g} +
    \delta_{gg'}\left(\hat\Sigma_{tg'}-\Sigma_{tlg'}\right)\right]\phi_{lg'}(z) +
    q(\mu,z,
\end{equation}
where $\hat\Sigma_{tg}$ is yet to be defined.
Also, it is assumed that the spectrum of energies leaving the sphere of \iso{U}{235} is well
approximated by the fission emission energy spectrum $\chi(E)$ (i.e., only consider the
uncollided energy spectrum).  The group integrated $\chi(E)$ for \iso{U}{235} is
computed.  All groups have a high scattering ratio of $\approx0.9999$.  Thus, a
few MFP away from the boundary source, we expect diffusion theory to be applicable,
with essentially a pure scatter. Consider the 1-speed diffusion theory equation for a pure
scatterer, in a semi-infinite medium, with no internal source
\begin{equation}
    -D \frac{\d^2\phi}{\d x^2} = 0.
\end{equation}
The requirement that the flux be bounded at infinity requires a constant spatial
solution, so we thus can neglect spatial gradients in Eq.~\eqref{te_orig}.
We choose to normalize the effective source of neutrons from the sphere such that the
magnitude of the energy integrated source is 1.  Thus, our transport equation to be solved becomes
\begin{equation}
    \hat\Sigma_{t,g}\psi_g(\mu) = \sum_{l=0}^\infty
    \frac{2l+1}{2} P_l(\mu) \sum_{g'=0}^{G-1}\left[\Sigma_{slg'\ra g} +
    \delta_{gg'}\left(\hat\Sigma_{tg'}-\Sigma_{tlg'}\right)\right]\phi_{lg'} +
    \frac{\chi_g}{2}
\end{equation}
Taking the zeroth moment gives the equation for the scalar flux in each group as
\begin{equation}
    \hat\Sigma_{t,g}\phi_g = \sum_{g'=0}^{G-1}\left[\Sigma_{s0g'\ra g} +
    \delta_{gg'}\left(\hat\Sigma_{tg'}-\Sigma_{t0g'}\right)\right]\phi_{g'} +
    \chi_g
\end{equation}
The first moment gives the equation for the current in each group as
\begin{equation}
    \hat\Sigma_{t,g}J_g = \sum_{g'=0}^{G-1}\left[\Sigma_{s1g'\ra g} +
    \delta_{gg'}\left(\hat\Sigma_{tg'}-\Sigma_{t1g}\right)\right]J_g'
\end{equation}
For the P$_1$ consistent form of the equation, we set $\hat\Sigma_{t,g} =
\Sigma_{t0,g}$, which simplifies the equation for the scalar flux to
\begin{equation}
    \Sigma_{t0,g}\phi_g = \sum_{g'=0}^{G-1}\Sigma_{s0g'\ra g}\phi_{g'} +
    \chi_g
\end{equation}





    



\end{solnum}


%\includepdf[pages={1}]{p1p3.pdf}
\clearpage
    \includepdf[pages={1-2}]{p3.pdf}
    A summary of the solutions obtained for each of the boundary conditions is given
    in the table below. The figures below compares plots for the case of $Q=1$,
    $D=0.5$ and $X=10$.  The first figure demonstrates that Dirichlet (without use of
    the extrapolated BC) can be very inaccurate.  The solution with Mark boundary
    conditions is lower in magnitude than for the Marshak case, which is expected due
    to symmetry and that the Mark boundary condition has a shorter extrapolation
    distance of $\sqrt(3)D$, compared to $2D$. The results of the boundary
    conditions very greatly for this case.  A smaller value of $D$ or increased $X$
    will result in more accurate solutions, relatively, on the interior of the
    domain. The second plot shows that albedo varies between a Marshak and Reflective
    condition.  As expected, the
    reflective BC on the far side results in a much larger magnitude in the solution
    as leakage is reduced.
    \begin{table}[h!]
        \centering
        \caption{Solutions with different boundary conditions for  a pure scatter for slab of width $X$ centered at
        $x=0$.}
        \begin{tabular}{|c|c|c|} \hline
            Left BC & Right BC & $\phi(x)$ \\ \hline
            Vac. Marshak & Vacuum Marshak & $\phi(x) = Q\left(\frac{X^2}{8D} + X-\frac{x^2}{2D} 
            \right)            $ \\ 
            Vac. Mark & Vacuum Marshak & $\phi(x) = Q\left(
            \frac{X^2}{8D} + \frac{X\sqrt{3}}{2}-\frac{x^2}{2D}\right)$ \\ 
            Vac. Dirichlet  & Vacuum Dirichlet & $\phi(x) = \frac{Q}{2D}\left( 
            \frac{X^2}{4} - x^2\right)  $ \\ 
            Vac. Dirichlet    & Albedo  & $\phi(x) = -\frac{Qx^2}{2D} +
            QxX\left(\frac{1+\A\frac{X}{2D}}{\A X + D} - \frac{1}{2D}  \right) +
            Q\frac{X^2}{2}\!\! \left(\frac{1+\A\frac{X}{2D}}{\A X +
            D}-\frac{1}{4D}\right) $ \\
            Vac. Dirichlet    & Reflecting & $\phi(x) =
            \frac{Q}{2D}\left(\frac{3X^2}{4} + xX  - {x^2}\right)$ \\ \hline
        \end{tabular}
    \end{table}
    \begin{figure}[h!]
        \centering
        \begin{subfigure}{0.495\textwidth}
            \centering 
            \includegraphics[width=0.99\textwidth]{diff_soln1.pdf}
        \end{subfigure}
        \begin{subfigure}{0.495\textwidth}
            \centering 
            \includegraphics[width=0.99\textwidth]{diff_soln2.pdf}
        \end{subfigure}
        \caption{Comparison of Diffusion solutions for various boundary conditions
        with Q=1, D=0.5, X=10.}
    \end{figure}
\clearpage
\subsubsection*{Code for Problem 1}
\lstinputlisting[basicstyle=\scriptsize]{group_collapse/cxproc.py}


\end{document}

