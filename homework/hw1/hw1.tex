% --------------------------------------------------------------
% This is all preamble stuff that you don't have to worry about.
% Head down to where it says "Start here"
% --------------------------------------------------------------
 
\documentclass[12pt]{article}

\usepackage{courier}
\usepackage{color}
\usepackage{listings}
\usepackage[square,numbers]{natbib}
\usepackage{tabls}
\usepackage{graphicx}
\usepackage{subcaption}
\usepackage{pdfpages}
\usepackage{mathtools}

\definecolor{dkgreen}{rgb}{0,0.6,0}
\definecolor{gray}{rgb}{0.5,0.5,0.5}




\lstset{language=Matlab,
   keywords={break,case,catch,continue,else,elseif,end,for,function,
      global,if,otherwise,persistent,return,switch,try,while},
   basicstyle=\ttfamily,
   keywordstyle=\color{blue},
   commentstyle=\color{red},
   stringstyle=\color{dkgreen},
   numbers=left,
   numberstyle=\tiny\color{gray},
   stepnumber=1,
   numbersep=10pt,
   backgroundcolor=\color{white},
   tabsize=4,
   showspaces=false,
   showstringspaces=false}
 
\usepackage[margin=1in]{geometry} 
\usepackage{amsmath,amsthm,amssymb}
\usepackage{verbatim}
\usepackage{algpseudocode,algorithm}
\usepackage{setspace}

\newcommand{\ihat}{\ensuremath{\hat{\textbf{\i}}}}
\newcommand{\jhat}{\ensuremath{\hat{\textbf{\j}}}}
\newcommand{\lline}{\noindent\makebox[\linewidth]{\rule{\textwidth}{0.4pt}}}
\newcommand{\N}{\mathbb{N}}
\newcommand{\Z}{\mathbb{Z}}
\newcommand{\deriv}[2]{\frac{\mathrm{d} #1}{\mathrm{d} #2}}
\newcommand{\pderiv}[2]{\frac{\partial #1}{\partial #2}}
\newcommand{\bx}{\mathbf{X}}
\newcommand{\ba}{\mathbf{A}}
\renewcommand{\d}{\mathrm{d}}
\newcommand{\upl}{u_{\text{plane}}}
\newcommand{\upt}{u_{\text{point}}}
\newcommand{\D}{\Delta}
\newcommand{\ra}{\rightarrow}
\renewcommand{\SS}{\State}
 
\newenvironment{theorem}[2][Theorem]{\begin{trivlist}
\item[\hskip \labelsep {\bfseries #1}\hskip \labelsep {\bfseries #2.}]}{\end{trivlist}}
\newenvironment{lemma}[2][Lemma]{\begin{trivlist}
\item[\hskip \labelsep {\bfseries #1}\hskip \labelsep {\bfseries #2.}]}{\end{trivlist}}
\newenvironment{exercise}[2][Exercise]{\begin{trivlist}
\item[\hskip \labelsep {\bfseries #1}\hskip \labelsep {\bfseries #2.}]}{\end{trivlist}}
\newenvironment{problem}[2][Problem]{\begin{trivlist}
\item[\hskip \labelsep {\bfseries #1}\hskip \labelsep {\bfseries #2:}]\hspace{0.3in}\newline\newline}{\end{trivlist}}
\newenvironment{question}[2][Question]{\begin{trivlist}
\item[\hskip \labelsep {\bfseries #1}\hskip \labelsep {\bfseries #2.}]}{\end{trivlist}}
\newenvironment{corollary}[2][Corollary]{\begin{trivlist}
\item[\hskip \labelsep {\bfseries #1}\hskip \labelsep {\bfseries #2.} ]}{\end{trivlist}}
\newenvironment{problem*}[1][Problem]{\begin{trivlist}
\item[\hskip \labelsep {\bfseries #1} {\hspace{-0.2em}\bfseries:}]}{\end{trivlist}}
\newenvironment{solution}[1][Solution]{\begin{trivlist}
\item[\hskip \labelsep {\bfseries #1} {\hspace{-0.2em}\bfseries:}]\hspace{0.3in}\newline}{\end{trivlist}}
 
\begin{document}
 
% --------------------------------------------------------------
%                         Start here
% --------------------------------------------------------------
 
\title{Homework 1}%replace X with the appropriate number
\author{Simon Bolding\\ %replace with your name
NUEN 629} %if necessary, replace with your course title
 
\maketitle

\clearpage

\includepdf[pages={1-3}]{Homework1.pdf}

%\includepdf[pages={1}]{p1p3.pdf}

\begin{problem}{1}
Henyey and Greenstein (1941) introduced a function which, by the variation of one
parameter, $−1 \leq h \leq 1$, ranges from backscattering through isotropic scattering to forward scattering. In
our notation we can write this as
\begin{equation}
    K( \mu_0 , v'\rightarrow v) = \frac{1}{2}
    \frac{1-h^2}{\left(1+h^2-2h\mu_0\right)^{3/2}}\delta(v'-v).
\end{equation}
Verify that this is a properly normalized $f ( \mu_0 )$ and compute $K_l (v'
\rightarrow → v)$ for $l = 0, 1, 2$ as a function of $h$.

\end{problem}

\begin{solution}
\includepdf[pages={1-2}]{p1.pdf}
\end{solution}
\clearpage

\begin{problem}{2}
In an elastic scatter between a neutron and a nucleus, the scattering angle in the center of mass system is
related to the energy change as
\begin{equation}\label{eq:E}
 \frac{E}{E'} = \frac{1}{2}\left((1+\alpha) + (1-\alpha)\cos \theta_c\right)
\end{equation}
where $E$ is the energy after scattering and $E'$′ is the initial energy of the neutron and
\begin{equation}
\alpha = \frac{(A-1)^2}{(A+1)^2}.
\end{equation}
The scattered angle in the center-of-mass system is related the lab-frame scattered angle as
\begin{equation}\label{eq:tan}
\tan \theta_L = \frac{A\sin \theta_c}{1 + A\cos \theta_c}
\end{equation}
Also, the distribution of scattered energy is given by
\begin{equation} \label{pdf}
P(E'\rightarrow E) = \left\{\begin{matrix}
\frac{1}{(1-\alpha)E'} & E'\alpha \leq E \leq E' \\ 
 0 & \text{otherwise}
\end{matrix}\right. .
\end{equation}
Derive an expression for $K( \mu_0, E'\rightarrow E)$, where $\mu_0$ is $\cos \theta_L$. What is the distribution in angle of neutrons of energy
in the range [0.05 MeV, 10 MeV] to energies below 1 eV if the scatter is with hydrogen?

\end{problem}

\begin{solution}

    \subsubsection*{Scattering Kernel Derivation}

Due to Eq.~\eqref{eq:E}, for a fixed $A$, a given value of $E$ and $E'$ fully define
$\mu_c$; the lab frame cosine of the scattering angle $\mu_0$ is also fully defined through
Eq.~\eqref{eq:tan}.  As a result, the shape of the doubly differential scattering cross section
 is fully defined by the probability density function (PDF) $P(E'\ra E)$. Thus, it is
possible to write the scattering cross section in the COM frame as~\cite{dunnshultis}
\begin{equation}
    \Sigma_s(\mu_0,E'\ra E) = \Sigma_s(E')P(E' \ra E) \delta(\mu_c - f_\mu(E,E'))
 \end{equation}
 where $f_\mu(E,E')$ is the value of $\mu_c$ that satisfies Eq.~\eqref{eq:E} for a given
 $E$, i.e.,
 \begin{equation}
     f_\mu(E,E') = \frac{2(\frac{E}{E'}) - (1+\alpha)}{(1-\alpha)}
 \end{equation}
Because we are interested in the scattering kernel as a function of the lab frame
cosine $\mu_0$, 
we define the scattering cross section in an equivalent form 
 \begin{equation}
     \Sigma_s(\mu_0,E'\ra E) = \Sigma_s(E')P(\mu_0)\delta(E - f_{E}(\mu_c(\mu_0),E'))
 \end{equation}
 where $P(\mu_0)$ is a PDF for $\mu_0$ given a certain value of $E'$, $f_E$ is
 defined as
 \begin{equation}
  f(\mu_0,E') = \frac{E'}{2}\left((1+\alpha) + (1-\alpha)\mu_c\right),
 \end{equation}
 and $\mu_c$ as a function of $\mu_0$ will be derived later in Eq.~\eqref{eq:muc}. The scattering kernel is defined as
 \begin{equation}
     K(\mu_0,E'\ra E) = \frac{\Sigma_s(E'\ra E,\mu_0)}{\displaystyle\int\limits_{0}^\infty\d
     E\!\int\limits_{-1}^1 \d \mu_0\,
 \Sigma_s(E'\ra E,\mu_0)}
 \end{equation}
 The denominator is evaluated as
 \begin{equation}
     \int\limits_{-1}^1 \!\d \mu_0 \int\limits_0^\infty \!\d E \;
     \Sigma_s(E')P(\mu_0)\delta(E-f_{E}(\mu_c(\mu_0),E)) =
     \Sigma_s(E')\int\limits_{-1}^1 \d\mu_0 P(\mu_0) = \Sigma_s(E')
 \end{equation}
 where the first equality is true because the argument of the delta function is zero
 for the value of $\mu_0$ and $E'$ that satisfy $f$, which in this case gives the
 $\mu_0$ that is the integration variable of the
 outer integral.  The scattering Kernel is then just
 \begin{equation}
     K(\mu_0,E'\ra E) = P(\mu_0)\delta(E - f_{E}(\mu_C(\mu_0),E)).  
 \end{equation}
 We now need to transform the PDF $P(E'\ra E)$ into a density function
 $P(\mu_0)$.  From Eq.~\eqref{eq:E},
 there is a one-to-one relationship between $E$ and $\mu_c=\cos(\Theta_c)$ in the
 range of $E\in[\alpha E',E']$, thus
 \begin{equation}
  P(E'\ra E) \d E = P(\mu_c)\d \mu_c
 \end{equation}
or
 \begin{equation} \label{pdfmu}
     P(\mu_c) = P(E' \ra E) \frac{\d E}{\d \mu_c} .
 \end{equation}
 Multiplication of Eq.~\eqref{eq:E} by $E'$, followed by differentiation, yields
 \begin{equation}
     \frac{ \d E}{\d \mu_c} = \frac{1}{2}(1-\alpha) E'
 \end{equation}
 Evaluating $\mu_c$ for $E$ at the limits $\alpha E'$
 and $E'$ gives the support for $P(\mu_c)$, defined for $\mu_c \in[ -1,1 ]$.  
 Substitution of the above equation and Eq.~\eqref{pdf} into Eq.~\eqref{pdfmu} gives
 the PDF in the COM frame
 \begin{equation}\label{eq:pdfc}
     P(\mu_c) = \frac{1}{(1-\alpha)E'} \left( \frac{1}{2}(1-\alpha)E' \right) =
     \frac{1}{2}, \quad \mu_c \in [-1,1]
 \end{equation}
 We must now transform to the lab frame scattering cosine $\mu_0$.  First, we solve
 Eq.~\eqref{eq:tan} for $\mu_0$ in terms of $\mu_c$ as follows:
 \begin{align}
  \tan^2\theta_L &= \left(\frac{A\sin \theta_c}{1 + A\cos \theta_c}\right)^2 \\
  \sec^2\theta_L - 1&= \left(\frac{A\sin \theta_c}{1 + A\cos \theta_c}\right)^2\\
  \mu_0^{-2} &= \frac{A^2(\sin^2\theta_c + \cos^2\theta_c) + 1 + 2A\mu_c}{\left(1+A\mu_c\right)^2}\\
  \mu_0 &= \frac{1+A\mu_c}{\sqrt{1+2\mu_cA+A^2}}. \label{eq:mul}
 \end{align}
Solution of the above equation for $\mu_c$ in terms of $\mu_L$ gives
\begin{equation}\label{eq:muc}
    \mu_c = -\frac{1}{A}(1-\mu_0^2) + \mu_0\sqrt{1-\frac{1}{A^2}(1-\mu_0^2)}\;.
\end{equation}
Eq.~\eqref{eq:mul} demonstrates a one-to-one relationship between $\mu_0$ and
$\mu_C$.  As before,
\begin{equation}\label{eq:pdfle}
P(\mu_0) = P\left(\mu_C(\mu_0)\right)\frac{\d \mu_c}{\d \mu_0}.
\end{equation}
Differentiation of Eq.~\eqref{eq:muc} with respect to $\mu_0$ and algebraic
manipulation ultimately yields
\begin{equation}
    \frac{\d \mu_c}{\d \mu_0} = \frac{2 \mu_0}{A} + \frac{1 - \frac{1}{A^2}(1 -
2\mu_0^2)}{\sqrt{1 - \frac{1}{A^2}(1 - \mu_0^2)}}.
\end{equation}
Substitution of the above equation and Eq.~\eqref{eq:pdfc} into Eq.~\eqref{eq:pdfle}
gives an expression for $P(\mu_0)$. The final expression for the scattering kernal
is, for $A>1$
\begin{equation}
    \boxed{
K(\mu_0,E'\rightarrow E) = \left\{\begin{matrix} \displaystyle
\left[\frac{\mu_0}{A} + \frac{1 - \frac{1}{A^2}(1 -
2\mu_0^2)}{2\sqrt{1 - \frac{1}{A^2}(1 -
\mu_0^2)}}\right]{\delta(E-f_E\left(\mu_c(\mu_0),E'\right))}, & \mu_0\in[-1,1] \\ 
 0, & \text{otherwise}
\end{matrix}\right. 
}.
\end{equation}
where the support is from evaluation of Eq.~\eqref{eq:mul} at $\mu_c=-1,1$.  The case
of $A=1$ must be treated separately. This can be seen, for instance, because evaluation of
Eq.~\eqref{eq:mul} at $\mu_c=-1$ results in an indeterminant $0/0$. Evaluation of
Eq.~\eqref{eq:muc} for $A=1$ gives a non-indeterminant expression for $\mu_0$ as
\begin{equation}\label{eq:mu0}
    \mu_0 = \sqrt{\frac{1+\mu_c}{2}}
\end{equation}
Thus, the support becomes $\mu_0 \in [0,1]$.  The kernel also simplifies significantly at
$A=1$.  The final scattering kernel, for the case of $A=1$, is
\begin{equation}\label{answer}
\boxed{
K(\mu_0,E'\rightarrow E) = \left\{\begin{matrix}
2\mu_0\,\delta(E-f_E\left(\mu_c(\mu_0),E'\right)), & \mu_0\in[0,1] \\ 
 0, & \text{otherwise}
\end{matrix}\right. 
}.
\end{equation}
which is PDF normalized over $\mu_0$ and $E$.

\subsubsection*{Plots for A=1}

To plot in terms of energy, we evaluate Eq.~\eqref{eq:E} at $A=1$, giving
\begin{equation}
    \frac{E}{E'} = \frac{1+\mu_c}{2}.
\end{equation}
Then, using Eq.~\eqref{eq:mu0}, $\mu_0$ in terms of $E$ and $E'$ is 
\begin{equation}
    \mu_0 = \sqrt{\frac{E}{E'}}.
\end{equation}
A plot of $P(\mu_0)$ vs $\mu_0$ and $P(\mu(E',E))$ vs $E'$ are given below.  The
values of $E'$ range between
$0.05$ MeV to $10$ MeV, with $E$ fixed at 1 eV.  As expected, lower energy neutrons
are more likely to scatter to 1 eV.
\begin{figure*}[hb]
    \includegraphics[width=0.5\textwidth]{scat_kernel.pdf}
    \includegraphics[width=0.5\textwidth]{scat_kernel_E.pdf}
\end{figure*}

 




 \end{solution}

 \begin{thebibliography}{9}

     \bibitem{dunnshultis}
         W.L. Dunn and J.K. Shultis, \emph{Exploring Monte Carlo Methods}, 2012.


\end{thebibliography}

\begin{problem}{3}
The problem details are given on the second page.
\end{problem}

\begin{solution}

\subsubsection*{Description of code}

The angular flux $\psi$ is computed by tracing characteristics as discussed in class.  To
compute points of intersection, the ray and surfaces of intersection are written in
parametric form. The position of a particle in the projected $x-y$ plane is denoted
$\mathbf{r} = x\ihat +
y\jhat$. Since we want to trace upstream, the parametric equation for the particle
position is given by 
\begin{equation}
    \mathbf{r} = (x_{i-1}-\Omega_xs)\ihat + (y_{i-1} - \Omega_ys)\jhat
\end{equation}
where $\mathbf{r}_{i-1}$ is the previous location, $s$ is a parameter that corresponds
to the signed distance the particle has traveled, and
\begin{align}
    \Omega_x =& \sin(\theta)\cos(\phi) \\
    \Omega_y =& \sin(\theta)\sin(\phi).
\end{align}
The parametric equation for each of the surfaces in the problem as a function of $x$ and
$y$ are given in Table~\ref{surfs}.  These equations are then evaluated with
$x=x_{i-1}-\Omega_xs$ and $y=y_{i-1}-\Omega_ys$ and solved for $s$ algebraically.   The smallest positive value of
$s$ from each equation, excluding surfaces where a solution does not exist,
corresponds to the next point of intersection.  If we were already at a surface, then
that solution will give $s=0$.  Care is taken to exclude this solution, accounting
for potential roundoff.
\begin{table}[h]
    \centering
    \caption{Parametric equations for surfaces in problem.\label{surfs}}
    \begin{tabular}{|c|c|c|} \hline
    Surface & $f(x,y)=0$ \\ \hline
    Fuel    & $x^2 + y^2 - R_{\text{fuel}}^2=0$ \\
    Left Boundary & $x - x_{\min}=0$ \\
    Right Boundary & $x - x_{\max}=0$ \\ 
    Bottom Boundary & $y - y_{\min}=0$ \\
        Top Boundary & $y - y_{\max}=0$ \\ \hline
    \end{tabular}
\end{table}

The particle is moved to the new location, and the number of mean free paths 
travelled $\tau_i=s_i\Sigma_{t}(x,y)$ along the $i$-th path is computed. The total number of MFP
traveled up to the latest point $s_i$ is accumulated as
$\tau_{\text{tot},i}=\sum_{k=1}^i \tau_k$. Because the transport equation is linear, we can consider the
contribution from each fuel element to the angular flux separately. If the $i$-th path
traced to point $\mathbf{r}_i$ was across a fuel element, then a contribution is made
to the flux.  If the path of length $s_i$ crossed the $j$-th fuel element, the contribution to the flux from that fuel element is
computed as
\begin{equation}
    \psi_j =
    \frac{Qe^{-\tau_{\text{tot},i-1}}}{4\pi\Sigma_{t,F}}\left(1-e^{-\Sigma_{t,F}s_i}\right)
\end{equation}
where $\tau_{\text{tot},i-1}$ does not include attenuation across the fuel because
that attenuation was accounted for in solution for the term in parenthesis. The
particle track is then continued from this point until.  The final solution for $\psi$, at
the location and direction of interest, will be 
\begin{equation}

\end{equation}

Finally, if the particle had hit a boundary, the corresponding coordinate is
translated to the opposing boundary,
taking care to handle roundoff issues and corners. For example, if the right boundary is hit at
point $\mathbf{r}=x_{\max}\ihat+y_1\jhat$, the particle is moved to $\mathbf{r} =
x_{\min}\ihat + y_1\jhat$.  The process outlined above is repeated for all directions
$\Omega$ and positions of interest.


\end{solution}


\end{document}



